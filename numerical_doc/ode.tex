\chapter{常微分方程式の求積}
\section{導入}
常微分方程式(Ordinary Differential Eqiations,ODEs)の問題は,
恒に1次の微分方程式の組を調べることに帰着する.
例えば2次の方程式,
\begin{equation}
 \frac{d^2 y}{dx} + q(x) \frac{dy}{dx} = r(x)
\end{equation}
は,1次の2つの方程式に書き換えられる.
\begin{eqnarray}
 \begin{cases}
  \cfrac{dy}{dx} = z(t) \\
  \cfrac{dz}{dt} = r(t)-q(t)z(t)
 \end{cases}
\end{eqnarray}
これは任意の常微分方程式に対しての例である.
ここで導入した新しい変数に対しての通常の選択は,単に互いと,もとの変数の導関数とすることである.
時には,オーバーフローや丸め誤差の増加を起こす特異な振る舞いを和らげる目的で,
その方程式のいくつかの因子や,いくつか独立変数の累乗のような定義を組み込むのは有用である,
前提として,解では変数は滑らかであるが,補助変数が異常な場合,それら理由の理解と違う補助変数を選択する.
常微分方程式での一般的問題は,N個の$y_i,i=1,2,\dots,N$に対しての連立1次微分方程式を調べることに帰着する.
\begin{equation}
 \frac{dy_i(t)}{dx} = f_i(x,y_i,\dots,y_N), \quad i=1,\dots,N \label{generical_ode}
\end{equation}
ここで右辺の関数$f_i$は解っている.

常微分方程式に関連する問題は,その方程式により完全に明記できない.
問題に対してどのように数値的に対するかを決定するのにより重要な事は,
問題の境界条件の性質である.

境界条件は式(\ref{generical_ode})で,関数$y_i(t)$の値の上の代数的条件である.
一般的に離散している特定の点で,条件は満たすことができる,しかし,それらの点の間の条件の保証はない.
例えば,微分方程式によって自動的に保存されない.
境界条件は単純に確定した数値を持っている確定した変数を要求できたり,複雑な変数間の非線形代数方程式の組にもできる.
通常,境界条件の本質は数値的方法で実行できるようなことで決定することである.
境界条件は2つカテゴリがある.
\begin{itemize}
 \item 初期値問題. すべての$x_s$での$y_i(x_s)$は与えられていて,終点$x_f$での$y_i(x_f)$の値を求める.
	   または,離散的な各点での値を求める.
 \item 2点境界問題. 1点より多い場所での値を指定する. 通常,$x_s$と$x_f$での値を指定する.
\end{itemize}